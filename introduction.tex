\specialsection{Введение}
За последние десятилетия огромную роль в изучении Вселенной стали играть исследования, основанные на глубоких и очень глубоких изображениях космических объектов. Связано это с техническим прогрессом и появлением телескопов, способных различить структуры, поверхностная яркость которых достигает 30 mag/arcsec$^2$. Возникновение такого большого количества наблюдательных данных создает огромный простор для новых исследований объектов, которые ранее были нам недоступны для наблюдений. 

В частности, одна наиболее интересная для изучения область неба, это полоса Sloan Digital Sky Survey (SDSS) Stripe 82 – полоса шириной 2.5 градуса вдоль небесного экватора с координатами -50\degree < R.A. < 60\degree, -1.25\degree< Dec < 1.25\degree и  с общей площадью 275 квадратных градусов во всех пяти фильтрах SDSS. Преимущества данной полосы заключаются в ее расположении: в первую очередь,  полоса Stripe 82 доступна для большинства наземных телескопов, что позволяет производить вспомогательные спектроскопические и фотометрические наблюдения. Во-вторых, эта полоса охватывает области от очень низкого, до высокого галактического поглощения, что может быть полезно для анализа рассеяния излучения пыли нашей Галактики. Конкретно для нашего исследования полоса Stripe 82 имеет преимущества в том, что в течение нескольких десятков лет в данной области были получены изображения неба порядка 80 раз во всех пяти SDSS фильтрах ugriz, в результате чего, мы имеет дело с кадрами, глубина которых отличается на 2 mag/arcsec$^2$ и более (для сравнения стандартные кадры SDSS достигают глубины 26.5 mag/arcsec$^2$ в фильтре r, тогда как средний предел по глубине изображений, загруженных из базы данных Stripe 82 в том же диапазоне, составляет 28.6 mag/arcsec$^2$). \footnote{\url{http://research.iac.es/proyecto/stripe82/}}

Глубина изображений в SDSS полосе Stripe 82 позволяет нам исследовать структуры низкой поверхностной галактик. Имея две выборки галактик с наблюдаемыми и отсутствующими приливными структурами мы можем сравнить средние толщины звездных дисков для обеих выборок (например, в работе \cite{1997A&A...324...80R} отношение радиального и вертикального масштабов $h_r/z_0$ примерно в 1.5-2 раза больше для галактик с приливными структурами, что можно объяснить большими дисперсиями скоростей звезд в вертикальном направлении у галактик, подверженных слиянию с другими объектами). Данная зависимость еще не является хорошо исследованной, поэтому в нашей работы мы попробуем пролить чуть больше света на вышеописанный вопрос.

Проектов, в рамках которых были обработаны изображения SDSS полосы Stripe 82 достаточно много. Один из них -- IAC Stripe 82 Legacy Project \footnote{\url{http://research.iac.es/proyecto/stripe82/}}. Основная цель проекта IAC Stripe 82 Legacy заключается в сохранении структур низкой поверхностной яркости во всех пространственных масштабах и масштабах интенсивности, используя неагрессивный метод вычитания фона неба. В нашей работе мы использем данные из IAC Stripe 82 Legacy Project, чтобы создать каталог галактик с ребра (далее ES82 – Edge-on galaxies in sdss Stripe 82) в полосе Stripe 82, с возможностью последующего исследования глубоких изображений этих объектов, выявления структур низкой поверхностной яркости (LSB – Low Surface Brightness), таких как полярные кольца, арки, хвосты и т.д., свидетельствующие о возможном взаимодействии галактик с их окружением, анализа параметров вертикальной и радиальной структуры звездных дисков спиральных галактик на основе данных о поверхностной фотометрии галактик, что может дать ответы на многие вопросы, связанные с образованием и эволюцией этих объектов. 




