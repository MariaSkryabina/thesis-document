\section{Полнота выборки}
С целью оценить полноту каталога нами был выполнен $V/V_m$ тест, описанный в работе \citep{1979ApJ...231..680T}. Мы расчитали для каждой галактики объем сферы \textit{V} с радиусом, равным расстоянию до объекта D, выражающимся следующим образом: $D = d/\theta$, где d -- линейный диаметр галактики, $\theta$ -- угловой. Также был посчитан объем сферы $V_m$, радиус которой, это максимальное расстояние $D_m$, на котором должна находиться галактика и всё еще принадлежать нашей выборке объектов. С учетом того, что $D_m = d/\theta_L$, данное расстояние будет соответствовать объекту выборки с наименьшим угловым размером, в нашем случае $\theta_L$ = 6.38 arcsec (за значение углового размера мы берем значение большой полуоси галактики \textit{SMA}). 
Отношение объемов $V/V_m =  (\theta_L/\theta)^3$. Среднее значение отношения этих объемов, которое мы будем обозначать как $\langle V/V_m \rangle$ должно равняться 0.5 для объектов равномерно распределенных в Евклидовом прострастве в случае, если выборка полная. 

Вычисляя это значение для каталога объектов, последовательно исключая объекты с маленькими угловыми размерами мы приходим к тому, что для $SMA>18$ arcsec (55$\%$ выборки) наша выборка, по существу, полна: в фильтре \textit{r} $\langle V/V_m \rangle = 0.489$.
