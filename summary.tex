\section{Заключение}
\begin{itemize}
    \item В рамках данной работы был создан каталог галактик с ребра в SDSS полосе Stripe 82 -- \textit{SG82.} Каталог на данный момент содержит 831 объект, включая галактики с углом наклона меньше 85\degree (визуально), 710 объектов, если исключить галактики, не являющиеся галактиками с ребра.
    \item Был проведен тест $V/V_m$ на полноту выборки. Для галактик с размером большой полуоси $SMA>18$ arcsec наша выборка, по существу, полна: в фильтре \textit{r} $\langle V/V_m \rangle = 0.489$.
    \item Были уточнены такие параметры галактик из каталога, как размеры полуосей и позиционные углы.
    \item Обработаны изображения объектов из каталога в фильтрах g, r, i, rdeep. 
    \item Получены цветные RGB и суммарные изображения.
    \item Отобрано и проклассифицировано 43 галактики, обладающие приливными LSB структурами, 86 галактик обладает структурными особенностями, не являющимися приливными. Важный результат этой работы заключается в том, что несмотря на глубину исследуемых изображений, количество галактик, обладающих структурами низкой поверхностной яркости в выборке намного меньше, чем ожидалось. Возможно глубины исследуемых данных недостаточно, чтобы выявить все галактики с приливными структурами и структурными особенностями.
    \item Выполнена одномерная фотометрическая декомпозиция для 73 галактик, для 51 галактики результаты фотометрической декомпозиции оказались хорошими или удовлетворительными. Оставшиеся 22 галактики либо не являются галактиками, видимыми точно с ребра, либо нуждаются в более сложных моделях для аппроксимации (например 3D модели с изломами диска). 
    \item Был произведен анализ масштабных параметров и сделаны оценки их средних значений. Средние значения вертикального и радиального масштабов, а также средние абсолютные звездные величины согласуются с результатами ранее выполненных работ других авторов. На основании графиков еще раз был подтвержден факт о наличии корреляции между радиальным и вертикальным масштабами диска, 
    \item Для всех галактик с результатами анализа распределения яркости выполнены отождествления с опубликованными списками спектральных красных смещений. Среднее красное смещение галактик выборки оказалось равным $\approx$ 0.064.
    \item Показано, что звездные диски галактик с наблюдаемыми приливными структурами в среднем являются более толстыми, чем диски галактик, у которых эти структуры отсутствуют.

\end{itemize}

Составленный каталог галактик можно в дальнейшем использовать для более подробного анализа структур низкой поверхностной яркости. Полученные результаты находятся в согласии с современными представлениями об эволюции дисковых подсистем галактик. Результаты дипломной работы дают важную информацию о структуре взаимодействующих галактик и они могут быть использованы для дальнейшего исследования фотометрических параметров и эволюции подсистем. 
