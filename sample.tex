\section{Выборка}
После первичного отбора галактик с ребра выборка включала 1167 объектов из каталога Fliri J., Trujillo I. \cite{2016MNRAS.456.1359F}, полученного в рамках проекта IAC Stripe 82 Legacy Project при помощи Sourse Extractor (сокращенно: Sextractor), а также каталогов EGIS\cite{2014ApJ...787...24B} и Galaxy Zoo \cite{2017yCat..74613663H}.
Работа программы SExtractor сводится к тому, что на подаваемом на вход
изображении в фитс-формате (fits) находятся объекты и вычисляются
некоторые их параметры, такие как размер, видимая звездная величина, эллиптичность, позиционный угол и пр. 
Для отбора объектов в каталог использовались следующие параметры SExtractor: $A\_IMAGE$, $B\_IMAGE$ -- большая и малая полуоси галактики, $KRON\_RADIUS$ -- радиус Крона. 
Критерии для отбора галактик в каталок:
\begin{itemize}
    \item Угловой размер галактики (радиус 25 изофоты $r_{25}$ -- радиус изофоты на котором угловая поверхностная яркость галактики достигает  $25 mag/arcsec^2$) не менее 15 arcsec. Данный параметр нам изначально не известен, но с помощью базы данных LEDA\footnote{\url{http://leda.univ-lyon1.fr/}} мы можем узнать размер $r_{25}$  для некоторых галактик из выборки и найти коэффициенты взаимосвязи радиуса Крона $KRON\_RADIUS$, известного параметра и $r_{25}$, тем самым усовершенствовав выборку объектов.
    \item Сжатие галактики ($B\_IMAGE/A\_IMAGE$) менеее 0.3
\end{itemize}

После были скачаны jpg изображения для полученного списка галактик и проведен визуальный отбор, в результате которого были удалены из выборки галактики, явно не относящиеся к галактикам, видимым с ребра: галактики, находящиеся под углом наклона менее 80-85 градусов (те, у которых видна плоскость диска, возможные спирали. В случае галактики, видимой точно с ребра, ветви не должны быть видны, в то время как даже при небольшом наклоне они могут стать видимыми), другие объекты.
Также проведенный визуальный анализ цветных RGB изображений позволил сузить список изучаемых объектов до 831 (важно отметить, что в данной выборке не все галактики являются галактиками с ребра, часть  объектов, не смотря на угол наклона < 85º, оставлены по причине наличия интересных структурных особенностей). Каталог, который включает только галактики с ребра состоит из 710 объектов.

В процессе работы были вручную переопределены размеры и позиционные углы у $40\%$ галактик, так как их изначальные значения были далеки от реальности.
Визуальный анализ цветных и контрастных изображений позволил сделать выводы о небольшом количестве галактик в полосе Stripe 82, обладающих структурами низкой поверхностной яркости (подробнее в разделе "Статистика проклассифицированных структур").  