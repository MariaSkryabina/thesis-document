\section{Статистика проклассифицированных структур}

Один из важных результатов данной работы, это статистика по распределению различных приливных структур, а также структурных особенностей галактик выборки.

Суммарно, приливные структуры низкой поверхностной яркости наблюдаются у 43 объектов. 
Количество галактик с различными приливными LSB структурами:
\begin{enumerate}
    \item Хвосты -- 6
    \item Мосты -- 8
    \item Арки -- 7
    \item Полярные кольца -- 2
    \item Остатки спутников -- 9
    \item Оболочки  -- 15
\end{enumerate}

Также собрана статистика по наблюдаемым структурам низкой поверхностной яркости, не являющихся прилиливными.
Всего галактик, обладающих подобными структурами – 86. Подробнее  по каждой структуре:
\begin{enumerate}
    \item Балджи (толстые коробкоподобные балджи, уплощенные балджеподобные структуры, полярные балджи)  --  36
    \item Изгибы диска  --  58
    \item Кособокость  --  10
    \item Другие (структурные особенности по типу спиралей)  --  5
\end{enumerate}

Основной результат данного раздела заключается в том, что несмотря на большую выборку галактик и глубину изображений, относительное количество объектов, обладающих приливными структурами, меньше ожидаемого. Взаимодействующие галактики, это распространенное явление во Вселенной, что мы можем наблюдать даже на примере нашей галактики Млечный Путь. 

