\section{Основные этапы работы}
В процессе работы были выделены следующие этапы ее выполнения:
\begin{enumerate}
    \item Отбор потенциальных галактик видимых с ребра, используя каталог объектов, полученный при помощи SExtractor  в работе Fliri J., Trujillo I \cite{2016MNRAS.456.1359F}. в рамках проекта IAC Stripe 82 Legacy Project, а также каталогов EGIS \cite{2017MNRAS.465.3784B} и Galaxy Zoo \cite{2017yCat..74613663H} (критерии отбора: угловой размер галактики, диаметр 25 изофоты,  не менее 30 arcsec, сжатие галактики $\frac{b}{a}$ менее 0.3).
    \item Проведение визуального отбора и усовершенствование выборки.
    \item Обработка изображений галактик в фильтрах g, r, i, rdeep (разворот, обрезка).
    \item Создание цветных RGB изображений.
    \item  Совершенствование выборки на основании полученных RGB изображений.
    \item Переопределение размеров галактик для более точной обрезки изображений (выполнено при помощи SAOImageDS9).
    \item Создание масок (для изображений в фильтрах rdeep, r)\footnote{Изображения-маски имеют размер как у изображения с исследуемой галактикой, но на нем помечены лишние объекты, которые попали на изображение. В процессе создания суммарных изображений или фотометрической декомпозиции не должны участвовать пиксели, которые заведомо
фоном неба не являются, их следует замаскировать.}, которые будут использоваться для создания суммарных изображений и фотометрической декомпозиции. 
    \item Создание суммарных изображений.
    \item Отбор галактик, обладающих структурами низкой поверхностной яркости (хвосты, арки, мосты, полярные кольца, диффузные оболочки и др.) на основании контрастных, цветных RGB изображений, изображений из обзора Legacy\cite{2019AJ....157..168D}, HSC-SSP (Hyper Suprime-Cam Subaru Strategic Program)\cite{2022PASJ...74..247A}.
    \item Классификация объектов по различным структурам низкой поверхностной яркости.
    \item Приведение статистики проклассифицированных структур.
    \item Выполнение одномерной фотометрической декомпозиции с целью определения нулевого приближения модельных параметров галактик и использования их в последующем этапе -- двумерной фотометричксой декомпозиции.
    \item Выполнение двумерной фотометрической декомпозиции.
    \item Анализ ее результатов.
    % \item Выполнение двумерной фотометрической декомпозиции для изображений галактик с замаскированной пылевой полосой.
    % \item Анализ полученных на предыдущих этапах параметров (соотношение вертикального и радиального масштабов,  распределение галактик выборки, по звездным величинам, по красным смещениям, сравнение толщин звездных дисков для галактик с наблюдаемыми и ненаблюдаемыми приливными структурами и др.).

\end{enumerate}